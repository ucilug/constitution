\documentclass{article}

\usepackage{hyperref}
\usepackage{fullpage}

\title{Constitution of the Linux User Group at UCI}

\begin{document}

\maketitle

\section{Name of the Organization}

This organization shall be named \textbf{Linux User Group at UCI}, hereafter 
referred to as \textbf{LUG@UCI}.

\section{Purpose of the Organization}

This organization shall be wholly dedicated to the advancement of awareness of 
GNU/Linux and the promotion of the use of GNU/Linux based systems.

\section{Membership}

The only requirement for membership is participation.  This means regular or 
semi-regular attendance of meetings, workshops, etc...

% Rights

% Every member hs a vote for the purposes of making decisions.

% Withdrawl of members.  How does one quit?  Is it possible for a member to be 
% kicked out?

\subsection{Associate Member}

Associate members are members that are considered a non-university person for 
the purposes of university policies.  The requirements and rights of associate 
members are the same as for regular members, except...

% Note: Members from outside the university cannot hold office, vote, or assume 
% fiscal responsibility.

\section{Making Decisions}

% See https://collab.ucilug.org/p/making_decisions

\section{Assignment of Tasks and Responsibilities}

% Definition of tasks V.S. responsibilities?

% Assignment process.

\subsection{Mandated Responsibilities}

% Signers

\section{Meetings}

% General Meetings.

% Regular Installfests/workshops?

\section{Finances}

There are no membership fees.

% When we \emph{do} need funds, what do we do?

\section{Advisors}

% Who can be an advisor?
% How do we pick?
% What do they do?

\section{Amendments}

% How do we accept changes to this document?

\section{Signature}

\end{document}
