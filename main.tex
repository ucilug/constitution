\documentclass{article}

\usepackage{hyperref}
\usepackage{fullpage}

\title{Constitution of the Linux User Group at UCI}

\begin{document}

\maketitle

\section{Name of the Organization}

This organization shall be named \textbf{Linux User Group at UCI}, hereafter 
referred to as \textbf{LUG@UCI}.

\section{Purpose of the Organization}

This organization shall be wholly dedicated to the advancement of awareness of 
GNU/Linux and the promotion of the use of GNU/Linux based systems.

\section{Membership}

% Rights

% Every member has a vote for the purposes of making decisions.

% Withdrawl of members.  How does one quit?  Is it possible for a member to be 
% kicked out?

\subsection{Rights given to Active Members}

Each active member shall have one vote in the decision making process.
Each active member shall contribute to the number of members needed for a quorum.

\subsection{Gaining Active Membership Status}

An active member is a member who has been present at at least three meetings or
has been designated as such by other active members. The active member must also
not have been previously found ineligible.

\subsection{Loss of Active Memebership Status}

Any member who fails to attend at least one meeting each academic quarter
(excluding summer) or who has their status revoked by major decision making
process will lose their active membership status.

\subsection{Associate Member}

Associate members are members that are considered a non-university person for 
the purposes of university policies.  The requirements and rights of associate 
members are the same as for regular members, except as prohibited by university
policies. These prohibitions include, but are not limited to, holding official
office, voting (including counting in a quorum), and assuming fiscal responsibility.

% Note: Members from outside the university cannot hold office, vote, or assume 
% fiscal responsibility.

\section{Making Decisions}

% See https://collab.ucilug.org/p/making_decisions

All decisions that will be voted on require at least two-thirds of all active members
to be present. Decisions can be voted on either in person at a meeting or, at the
member's discretion, online in a specified medium. In the case of an online vote,
at least two-thirds of all active members must vote for the result to be considered valid.
Votes will be either 'Yes', 'No', or 'Abstain' (only to be used in situations of a significant
conflict of interest).

\subsection{Minor Decision Making Process}

All decisions designated as minor shall require a simple majority of the members present
to vote in favor of it for the decision to be official.

\subsection{Major Decision Making Process}

All decisions designated as major shall require a two-thirds majority of the members present
to vote in favor of it for the decision to be official.

\section{Assignment of Tasks and Responsibilities}

% Definition of tasks V.S. responsibilities?

% Assignment process.

\subsection{Mandated Responsibilities}

% Signers

\textbf{LUG@UCI} may periodically designate individuals with tasks.
Any temporary responsibilities will be determined and assigned at meetings
through the appropriate responsibilities. However, there are some persistant
responsibilities the organization requires, which include:

\begin{itemize}
\item Signer Penguins
\item Public Penguin
\item Media Goblin
\item Meeting Leading Penguin
\item Infrastructure Penguin
\end{itemize}

All responsibilities excepting that of Signers shall be assigned by minor
decision. Signers are three to five members assigned at the begining of each
academic year who shall be responible for the tasks designated to them by
the University, which include access to finacial reserves.

\section{Meetings}

% General Meetings.
\subsection{General Meetings}

\textbf{LUG@UCI} will hold a general meeting each week during the academic year,
excluding finals week. The exact day of meetings and any meetings during breaks
shall be determined through the appropriate decision making process.

% Regular Installfests/workshops?
\subsection{Installfests}

\textbf{LUG@UCI} may hold periodic installfests during week two of each
academic quarter. All participants should refer to the Installfest liability
form for general precautions at the event.

\section{Finances}

No membership dues shall be required to join \textbf{LUG@UCI} as a member.

Any and all expenditures and fundraising must be approved as a Major Decision.
A Signer shall then be designated to carry out the expenditure.

% When we \emph{do} need funds, what do we do?

\section{Advisors}

% Who can be an advisor?
% How do we pick?
% What do they do?

\subsection{Requirements for Advisors}

All advisors shall be current faculty of the University.

\subsection{Selection of Advisors}

Advisors shall be selected by the following method:

\begin{enumerate}
\item A list of potential advisors will be drafted as a minor decision.
\item Potential advisors from the list will be contacted for their
    interest and consent.
\item An advisor will be selected from the list of those from whom
    consent has been recieved through a Major Decision.
\end{enumerate}

\subsection{Responsibilities of Advisors}

The Advisor of \textbf{LUG@UCI} may offer advice, network with other faculty and
researchers, participate in decision making as an active member, and provide
addtional support in assisting the organization with locating a meeting location
and resolving a conflict with the University.

\section{Amendments}

% How do we accept changes to this document?

Any changes to this document require a submission of the proposed revisions to the
group as a whole. The changes will be ratified at a general meeting as a Major Decision
after adequate time is given for comments and revisions, which shall not be less than one
week's time.

\section{Ratification}

% How do we accept this document?

This document shall not have effect until ratified by the standards set forth for
a Major Decision.

\section{Signature}

\end{document}
